% !TEX root = hobbes.tex

\documentclass[svgnames,fleqn]{beamer}

\input ../usepackage
\input ../environment
\input ../macros

\begin{document}

\setbeamertemplate{headline}[default]
\setbeamertemplate{footline}[default]
\title{Hobbes}
\author{TGN}
\date{}
\frame\titlepage
\setcounter{page}{1}
\setbeamertemplate{footline}[text line]{
\parbox{\linewidth}{\hfill {
%{\color{palegrey}
Guangning Tan
%}
}\hfill{\large\insertpagenumber}
}}

\begin{frame}{Hobbes}

https://github.com/morganstanley/hobbes

\bi
\item High performance C/C++ binding
\item REPL: Read, Evaluate, Print, Loop
\item Similar to Python, but variables are strongly typed during ``compile time"
\bi
\item ... when the hobbes script is parsed
\ei
\item Take strings (of script) as input, and execute C/C++ code
\item Can modify code logic, without recompiling code
\ei

\end{frame}

\begin{frame}{Two executables: \li{hi} and \li{hog}}
\bi
\item \li{hi}: interactive hobbes
\item \li{hog}: to record structured data into files
\ei
\end{frame}

\begin{frame}{Syntax}
\bi
\item \li{\\\\Var1 Var2 ... VarN.Expression} \\
defines a function \li{Expression} in variables \li{Var1}, \li{Var2}, ... , \li{VarN}
\bi
\item Example:\\
\li{\\\\x y.x+y}\\
takes \li{x}, \li{y} as input and returns \li{x+y}
\ei
\item \li{::} reads ``has type"
\bi
\item Example:
\li{sensor::([char] * { temperature:double, humidity:double })}
means: \li{sensor} is a 2-element tuple, marked by \li{*}
\bi
\item the first element is a string
\item the second element is a structure with two doubles
\ei
\ei
\ei

\end{frame}

\end{document}